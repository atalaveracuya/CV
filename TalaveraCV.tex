%%%%%%%%%%%%%%%%%%%%%%%%%%%%%%%%%%%%%%%%%
% Este documento se basa en una plantilla disponible en:
% https://www.LaTeXTemplates.com
%
% Original author:
% Trey Hunner (http://www.treyhunner.com/)

%%%%%%%%%%%%%%%%%%%%%%%%%%%%%%%%%%%%%%%%%

%----------------------------------------------------------------------------------------
%	PACKAGES AND OTHER DOCUMENT CONFIGURATIONS
%----------------------------------------------------------------------------------------

\documentclass[
%a4paper, % Uncomment for A4 paper size (default is US letter)
11pt, % Default font size, can use 10pt, 11pt or 12pt
]{resume} % Use the resume class

\usepackage{ebgaramond} % Use the EB Garamond font
\usepackage{hyperref}
%------------------------------------------------

\name{Andrés Talavera Cuya} 

%----------------------------------------------------------------------------------------

\begin{document}	
	
	%----------------------------------------------------------------------------------------
	%	Acerca de mí
	%----------------------------------------------------------------------------------------
	
	\begin{rSection}{}
		Bachiller en Ciencias Económicas de la Universidad Nacional Federico Villarreal, perteneciente al décimo superior, con experiencia en la construcción y monitoreo de indicadores socioeconómicos. Manejo avanzado de programas para el análisis estadístico o econométrico: \textit{Stata, R, Python, Tableau, Power Bi, KoboCollect}, entre otros. Con conocimientos en la redacción de informes y estudios. Predisposición de desarrollo profesional en áreas de investigación y análisis económico.
		
	\end{rSection}
	
	%----------------------------------------------------------------------------------------
	%	Información de contacto
	%----------------------------------------------------------------------------------------
	
	\begin{rSection}{Información de contacto}
		
		Lugar y fecha de nacimiento	: Lima, Perú | 16 de enero de 1987 \\
		Dirección					: Jr. Jorge Chamot Biggs Mz A-17, S. de Surco - 
		Lima \\
		Teléfono					: +051 923 732 307 \\
		E-mail						: atalaveracuya@gmail.com 
	\end{rSection}
	
	%----------------------------------------------------------------------------------------
	%	Educación
	%----------------------------------------------------------------------------------------
	
	\begin{rSection}{Educación}
		
		\textbf{Universidad Nacional Federico Villarreal, Lima} \hfill \textit{Abr. de 2005 – dic. de 2010} \\ 
		Bachiller en Ciencias Económicas  \\
		Orden de mérito: décimo superior. Calificación: 15.24. Cuadro de méritos: puesto 13° de un total de 280 alumnos, por rendimiento académico en el periodo 2005-1 al 2010-2.
		
	\end{rSection}
	
	
	
	%----------------------------------------------------------------------------------------
	%	Ponencias y consultorías
	%----------------------------------------------------------------------------------------
	\begin{rSection}{Ponencias y consultorías}
		
		\item Ponencia Construcción de indicadores del Censo Nacional de Comisarías  
		\href{https://bit.ly/3JgRMjN}{(link)} \hfill \textit{Mzo. de 2022}  
		\item Ponencia Importancia de la Encuesta Permanente de Empleo del INEI \href{https://bit.ly/3ISHYuX}{(link)}  \hfill \textit{Dic de 2021} 
		\item Ponencia taller Encuesta Nacional de Hogares con Stata \href{https://bit.ly/3mpje65}{(link)} \hfill \textit{Ago. de 2021} 
		\item Consultoría – consolidado, etiquetado y análisis del IV Cenagro 2012 \hfill \textit{Jul. de 2019} 
		
	\end{rSection}
	
	%------------------------------------------------
	
	
	
	%----------------------------------------------------------------------------------------
	%	Experiencia laboral
	%----------------------------------------------------------------------------------------
	
	\begin{rSection}{Experiencia}

		\begin{rSubsection}{Analista}{09/08/2024 – 31/12/2024}{INEI - Dirección Nacional de Cuentas Nacionales}{}
			
			\item  Analizar y mejorar los resultados de la segunda vuelta en el módulo ERETES, de los equilibrios de los productos pertenecientes a las actividades económicas “Transporte acuático, Transporte aéreo y Correo y mensajería” con información de las importaciones, tasas de márgenes e impuestos y análisis del destino del consumo intermedio y demanda final, en el marco del Nuevo Año Base de las Cuentas Nacionales.
            \item  Preparar los datos para la tercera vuelta y análisis de la Cuenta de Producción y Generación del Ingreso y sus componentes, así como el empleo y los ratios de las actividades económicas “Transporte acuático, Transporte aéreo y Correo y mensajería”, en el marco del Nuevo Año Base de las Cuentas Nacionales.
            \item Evaluar y mejorar los equilibrios de la tercera vuelta de los productos pertenecientes a las actividades económicas “Transporte acuático, Transporte aéreo y Correo y mensajería” con información de las importaciones, tasas de márgenes e impuestos y análisis del destino del consumo intermedio y demanda final, en el marco del Nuevo Año Base de las Cuentas Nacionales.
            \item Elaborar los datos para la cuarta vuelta y análisis de la Cuenta de Producción y Generación del Ingreso y sus componentes, así como el empleo y los ratios de las actividades económicas “Transporte acuático, Transporte aéreo y Correo y mensajería”, en el marco del Nuevo Año Base de las Cuentas Nacionales.
            \item Revisar y mejorar los equilibrios de la cuarta vuelta de los productos pertenecientes a las actividades económicas “Transporte acuático, Transporte aéreo y Correo y mensajería” con información de las importaciones, tasas de márgenes e impuestos y análisis del destino del consumo intermedio y demanda final, en el marco del Nuevo Año Base de las Cuentas Nacionales
			
		\end{rSubsection}
		
		%------------------------------------------------

		\begin{rSubsection}{Asistente}{02/02/2024 – 03/08/2024}{INEI - Dirección Nacional de Cuentas Nacionales}{}
			
			\item Elaborar las cuentas de producción y generación del ingreso de los hogares 2019, por productos e insumos a partir del ingreso y gasto de las unidades productivas del trabajador independiente, según fuente ENAHO 2019 y revisión principalmente del valor bruto de producción y consumo intermedio de las actividades económicas Transporte acuático, Transporte aéreo y Correo y mensajería, en el marco del Nuevo Año Base de las Cuentas Nacionales.
			\item Elaborar y mejorar los Balances de Oferta y Utilización en coordinación con las otras actividades económicas para el Cuadro de Oferta y Utilización de las actividades económicas Transporte acuático, Transporte aéreo y Correo y mensajería. 
			\item Revisar los resultados del valor bruto de producción y consumo intermedio y valor agregado bruto, y los Balances de Oferta y Utilización de las actividades económicas Transporte acuático, Transporte aéreo y Correo y mensajería. 
			
		\end{rSubsection}
		
		%------------------------------------------------
		
		\begin{rSubsection}{Operador de empresa}{22/3/2021 – 19/6/2021}{INEI - Dirección Nacional de Censos y Encuestas}{}
			\item Realizar las llamadas a las empresas que pertenecen a la muestra para coordinar y orientar al contador, gerente o responsable de la empresa respecto al diligenciamiento del formulario electrónico de la Encuesta Económica Anual.
			\item Ingresar diariamente la información sobre la ocurrencia de las coordinaciones y llamadas a las empresas seleccionadas en el Sistema de información y hacer la entrega diaria de la carga de trabajo.
			\item Efectuar el seguimiento a las empresas omisas a través de la comunicación de las notificaciones a fin de persuadir al informante para que cumpla con el diligenciamiento del formulario electrónico de la Encuesta Económica Anual.
		\end{rSubsection}
		
		%------------------------------------------------
		
			
		\begin{rSubsection}{Analista de base de datos}{17/9/2019 – 2/2/2020}{INEI - Dirección Nacional de Censos y Encuestas}{}
			\item Elaboración de reglas lógicas y flujos de consistencia.
			\item Elaboración de sintaxis.
			\item Elaboración de reportes de calidad y monitoreo.
			\item Elaboración de sintaxis de indicadores.
		\end{rSubsection}
			
			%------------------------------------------------
			
			\begin{rSubsection}{Practicante profesional}{4/5/2015 - 3/3/2016}{Fondo Mivivienda S.A. Oficina de Planeamiento, Prospectiva y Desarrollo Organizacional (OPPD)}{}
				\item	Actualización de estadísticas del Fondo Mivivienda.
				\item	Apoyo en la elaboración de informes de gestión para la Superintendencia de Mercado de Valores (S.M.V.)
				\item	Asistir en la elaboración de presentaciones para conferencias de prensa y reportes para la Alta Gerencia, Ministerio de Vivienda Construcción y Saneamiento, así como entidades nacionales e internacionales (S.M.V., Superintendencia de Banca Seguros y A.F.P., Banco Interamericano de Desarrollo, Fitch Rating, entre otros).
				\item	Sistematización, limpieza y análisis de la base de datos de la Encuesta Nacional de Hogares usando Stata. 
				\item	Elaboración de mapas estadísticos usando Stata.
			\end{rSubsection}
			
			%------------------------------------------------
			
			\begin{rSubsection}{Asistente Estadístico}{1/3/2010 – 27/7/2011}{Encuestadora Gauss Data E.I.R.L.}{}
				\item	Apoyo en la elaboración de informes de resultados.
				\item	Tareas de consistencia, procesamiento y análisis de base de datos de encuestas a hogares.
			\end{rSubsection}
			
		\end{rSection}
		
		
		%----------------------------------------------------------------------------------------
		%	Actividades de docencia
		%----------------------------------------------------------------------------------------
		
		\begin{rSection}{Actividades de docencia}
			
			\begin{rSubsection}{Grupo CEA}{Febr. de 2021 – Oct. de 2021}{}{}
				\item	Módulo 1 – Programa de Especialización Master Program Stata (52 horas lectivas)
				\item	Programación Estadística con Stata (40 horas lectivas)
				\item	Indicadores Socioeconómicos con Stata (28 horas lectivas)
			\end{rSubsection}
			
		\end{rSection}
		
		%----------------------------------------------------------------------------------------
		%	Redacción y Normas APA 7ma Edición: Cursos y Talleres
		%----------------------------------------------------------------------------------------
		
		\begin{rSection}{Redacción y Normas APA 7ma Edición: Cursos y Talleres}
			\item Curso de Redacción Nivel Básico, Intermedio y Avanzado (70 horas académicas) \hfill \textit{Feb. de 2023} \ 
			\item Curso Taller en Uso de Microsoft Word para la Redacción de Textos Académicos (18 horas académicas) \hfill \textit{Ago. de 2022} \ 
			\item Taller Normas APA en 2 semanas (16 horas académicas) \hfill \textit{Dic. de 2021} \ 
			\item Redacción para Investigadores - Uso de Normas APA 7ma edición (30 horas académicas) \hfill \textit{Jul. de 2020}  
		\end{rSection}
		
		%----------------------------------------------------------------------------------------
		%	Informática
		%----------------------------------------------------------------------------------------
		
		\begin{rSection}{Informática}
			\item Reportes con R (24 horas) \hfill \textit{Expedición Mzo. de 2023} \
			\item R Studio para Ciencia de Datos Nivel Básico, Intermedio y Avanzado (75 horas) \hfill \textit{Exp. Mzo. de 2023} \ 
			\item Econometría con Stata (69 horas) \hfill \textit{Exp. Ene. de 2023} \
			\item Power Bi nivel básico, intermedio y avanzado (45 horas) \hfill \textit{Exp. Dic. de 2022}
			\item Tableau (15 horas) \hfill \textit{Exp. Jul. de 2022}
			\item Indicadores Económicos con Stata (33 horas) \hfill \textit{Exp. Ene. de 2022} 
			\item Econometría de Datos Panel (160 horas) \hfill \textit{Exp. Ago. de 2021}
			\item Econometría Básica con Python (153 horas) \hfill \textit{Exp. Mzo. de 2021}
			\item Elaboración de Indicadores Socioeconómicos con la Enaho usando Stata (24 horas) \hfill \textit{Exp. Nov. de 2020}
			\item Stata Intermedio (30 horas) \hfill \textit{Exp. Mzo. de 2020}
			\item Microsoft Excel avanzado (32 horas) \hfill \textit{Exp. Nov. de 2015}
			\item Métodos Estadísticos con SPSS (24 horas)  \hfill \textit{Exp. May. de 2010} \
			
		\end{rSection}
		
		
%----------------------------------------------------------------------------------------
		%	Voluntariado
		%----------------------------------------------------------------------------------------
		
		\begin{rSection}{Voluntariado}
			
			\textbf{ONG Techo Perú, Lima} \hfill \textit{16/8/2016 – 8/4/2019} \\ 
			Voluntario del área de Investigación Social. 
		\end{rSection}
		
		\begin{rSection}{Intereses}
			Participar en proyectos de investigación o publicaciones académicas. Voluntariado, lectura, programación y música.
		\end{rSection}	
		
		
	\end{document}
